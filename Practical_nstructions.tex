% Options for packages loaded elsewhere
\PassOptionsToPackage{unicode}{hyperref}
\PassOptionsToPackage{hyphens}{url}
%
\documentclass[
]{article}
\usepackage{amsmath,amssymb}
\usepackage{iftex}
\ifPDFTeX
  \usepackage[T1]{fontenc}
  \usepackage[utf8]{inputenc}
  \usepackage{textcomp} % provide euro and other symbols
\else % if luatex or xetex
  \usepackage{unicode-math} % this also loads fontspec
  \defaultfontfeatures{Scale=MatchLowercase}
  \defaultfontfeatures[\rmfamily]{Ligatures=TeX,Scale=1}
\fi
\usepackage{lmodern}
\ifPDFTeX\else
  % xetex/luatex font selection
\fi
% Use upquote if available, for straight quotes in verbatim environments
\IfFileExists{upquote.sty}{\usepackage{upquote}}{}
\IfFileExists{microtype.sty}{% use microtype if available
  \usepackage[]{microtype}
  \UseMicrotypeSet[protrusion]{basicmath} % disable protrusion for tt fonts
}{}
\makeatletter
\@ifundefined{KOMAClassName}{% if non-KOMA class
  \IfFileExists{parskip.sty}{%
    \usepackage{parskip}
  }{% else
    \setlength{\parindent}{0pt}
    \setlength{\parskip}{6pt plus 2pt minus 1pt}}
}{% if KOMA class
  \KOMAoptions{parskip=half}}
\makeatother
\usepackage{xcolor}
\usepackage[margin=1in]{geometry}
\usepackage{color}
\usepackage{fancyvrb}
\newcommand{\VerbBar}{|}
\newcommand{\VERB}{\Verb[commandchars=\\\{\}]}
\DefineVerbatimEnvironment{Highlighting}{Verbatim}{commandchars=\\\{\}}
% Add ',fontsize=\small' for more characters per line
\usepackage{framed}
\definecolor{shadecolor}{RGB}{248,248,248}
\newenvironment{Shaded}{\begin{snugshade}}{\end{snugshade}}
\newcommand{\AlertTok}[1]{\textcolor[rgb]{0.94,0.16,0.16}{#1}}
\newcommand{\AnnotationTok}[1]{\textcolor[rgb]{0.56,0.35,0.01}{\textbf{\textit{#1}}}}
\newcommand{\AttributeTok}[1]{\textcolor[rgb]{0.13,0.29,0.53}{#1}}
\newcommand{\BaseNTok}[1]{\textcolor[rgb]{0.00,0.00,0.81}{#1}}
\newcommand{\BuiltInTok}[1]{#1}
\newcommand{\CharTok}[1]{\textcolor[rgb]{0.31,0.60,0.02}{#1}}
\newcommand{\CommentTok}[1]{\textcolor[rgb]{0.56,0.35,0.01}{\textit{#1}}}
\newcommand{\CommentVarTok}[1]{\textcolor[rgb]{0.56,0.35,0.01}{\textbf{\textit{#1}}}}
\newcommand{\ConstantTok}[1]{\textcolor[rgb]{0.56,0.35,0.01}{#1}}
\newcommand{\ControlFlowTok}[1]{\textcolor[rgb]{0.13,0.29,0.53}{\textbf{#1}}}
\newcommand{\DataTypeTok}[1]{\textcolor[rgb]{0.13,0.29,0.53}{#1}}
\newcommand{\DecValTok}[1]{\textcolor[rgb]{0.00,0.00,0.81}{#1}}
\newcommand{\DocumentationTok}[1]{\textcolor[rgb]{0.56,0.35,0.01}{\textbf{\textit{#1}}}}
\newcommand{\ErrorTok}[1]{\textcolor[rgb]{0.64,0.00,0.00}{\textbf{#1}}}
\newcommand{\ExtensionTok}[1]{#1}
\newcommand{\FloatTok}[1]{\textcolor[rgb]{0.00,0.00,0.81}{#1}}
\newcommand{\FunctionTok}[1]{\textcolor[rgb]{0.13,0.29,0.53}{\textbf{#1}}}
\newcommand{\ImportTok}[1]{#1}
\newcommand{\InformationTok}[1]{\textcolor[rgb]{0.56,0.35,0.01}{\textbf{\textit{#1}}}}
\newcommand{\KeywordTok}[1]{\textcolor[rgb]{0.13,0.29,0.53}{\textbf{#1}}}
\newcommand{\NormalTok}[1]{#1}
\newcommand{\OperatorTok}[1]{\textcolor[rgb]{0.81,0.36,0.00}{\textbf{#1}}}
\newcommand{\OtherTok}[1]{\textcolor[rgb]{0.56,0.35,0.01}{#1}}
\newcommand{\PreprocessorTok}[1]{\textcolor[rgb]{0.56,0.35,0.01}{\textit{#1}}}
\newcommand{\RegionMarkerTok}[1]{#1}
\newcommand{\SpecialCharTok}[1]{\textcolor[rgb]{0.81,0.36,0.00}{\textbf{#1}}}
\newcommand{\SpecialStringTok}[1]{\textcolor[rgb]{0.31,0.60,0.02}{#1}}
\newcommand{\StringTok}[1]{\textcolor[rgb]{0.31,0.60,0.02}{#1}}
\newcommand{\VariableTok}[1]{\textcolor[rgb]{0.00,0.00,0.00}{#1}}
\newcommand{\VerbatimStringTok}[1]{\textcolor[rgb]{0.31,0.60,0.02}{#1}}
\newcommand{\WarningTok}[1]{\textcolor[rgb]{0.56,0.35,0.01}{\textbf{\textit{#1}}}}
\usepackage{longtable,booktabs,array}
\usepackage{calc} % for calculating minipage widths
% Correct order of tables after \paragraph or \subparagraph
\usepackage{etoolbox}
\makeatletter
\patchcmd\longtable{\par}{\if@noskipsec\mbox{}\fi\par}{}{}
\makeatother
% Allow footnotes in longtable head/foot
\IfFileExists{footnotehyper.sty}{\usepackage{footnotehyper}}{\usepackage{footnote}}
\makesavenoteenv{longtable}
\usepackage{graphicx}
\makeatletter
\def\maxwidth{\ifdim\Gin@nat@width>\linewidth\linewidth\else\Gin@nat@width\fi}
\def\maxheight{\ifdim\Gin@nat@height>\textheight\textheight\else\Gin@nat@height\fi}
\makeatother
% Scale images if necessary, so that they will not overflow the page
% margins by default, and it is still possible to overwrite the defaults
% using explicit options in \includegraphics[width, height, ...]{}
\setkeys{Gin}{width=\maxwidth,height=\maxheight,keepaspectratio}
% Set default figure placement to htbp
\makeatletter
\def\fps@figure{htbp}
\makeatother
\setlength{\emergencystretch}{3em} % prevent overfull lines
\providecommand{\tightlist}{%
  \setlength{\itemsep}{0pt}\setlength{\parskip}{0pt}}
\setcounter{secnumdepth}{-\maxdimen} % remove section numbering
\ifLuaTeX
  \usepackage{selnolig}  % disable illegal ligatures
\fi
\usepackage{bookmark}
\IfFileExists{xurl.sty}{\usepackage{xurl}}{} % add URL line breaks if available
\urlstyle{same}
\hypersetup{
  pdftitle={2024 ELLS summer school Darwinian agriculture},
  pdfauthor={Jacques David, Nicolas Salas, Peter Bourke},
  hidelinks,
  pdfcreator={LaTeX via pandoc}}

\title{2024 ELLS summer school Darwinian agriculture}
\usepackage{etoolbox}
\makeatletter
\providecommand{\subtitle}[1]{% add subtitle to \maketitle
  \apptocmd{\@title}{\par {\large #1 \par}}{}{}
}
\makeatother
\subtitle{Workshop Social evolution under mass selection or under true
breeding values}
\author{Jacques David, Nicolas Salas, Peter Bourke}
\date{2024-06-20}

\begin{document}
\maketitle

\section{Introduction}\label{introduction}

Welcome to this practical session on the decomposition of phenotypic
value and the effects of selection. In this session, we will explore how
the phenotypic value of a plant is influenced by its genotype and by the
genotypes of its neighbors, and how selection on these traits affects
the next generation.

\section{Key theoretical Concepts}\label{key-theoretical-concepts}

find the ref Bijma , Muir etc

\subsection{Decomposition of the Phenotypic
Value}\label{decomposition-of-the-phenotypic-value}

Classically, in one single macro-environment, the phenotypic value
(\(P\)) of an individual can be decomposed into multiple components:
\[ P = G + E \]\\
Where:\\
- \textbf{G} represents the genetic contribution to the phenotype,\\
- \textbf{E} represents the micro-environmental contribution to the
phenotype.

Let's rigorously derive the total phenotypic variance \(V_P\) where the
phenotypes may not independent, either because of genetic relatedness or
structured environment.

Consider the phenotype \(P_i\) of an individual \(i\), influenced by its
genetic effect (\(G_i\)) and its direct environmental effect (\(e_i\)):
\(P_i = G_i + e_i\)

\subsection{Non-Independent
Phenotypes}\label{non-independent-phenotypes}

Consider the phenotype \(P_i\) of an individual \(i\),
\(P_i = G_i + e_i\),

where \(G_i\) is the genetic effect and \(e_i\) is the environmental
effect.

For the sake of simplicity, let'use centered value of \(P\), with so
that \(E(P)=E(G)=E(e)=0\), and
\(V_P=E(P_i^2); V_G=E(G_i^2) ; V_e=E(e_i^2)\)

The total phenotypic variance is defined as:

\(V_P = E(P_i)^2\)

First, we need to determine \(E(P_i)\):

\$E(P) = \frac{1}{n} \sum\_\{i=1\}\^{}n P\_i \$

where \(n\) is the number of individuals.

\[\begin{split}
E(P_i-\frac{1}{n} \sum_{i=1}^n P_i )^2 &= E(P_i-\frac{1}{n} (P_i+..+P_j ))^2\\
&=E(P_i - \frac{1}{n}(P_iP_j + P_j + P_jP_i..))^2
\end{split}\]

Next, let's calculate \(E(P^2)\):

\(P^2 = \left( \frac{1}{n} \sum_{i=1}^n P_i \right)^2 = \frac{1}{n^2} \sum_{i=1}^n \sum_{j=1}^n P_i P_j\)

Taking the expectation:

\[ E(P^2) = \frac{1}{n^2} \sum_{i=1}^n \sum_{j=1}^n E(P_i P_j) \]

\paragraph{\texorpdfstring{Step 3: Calculate
\(E(P_i P_j)\)}{Step 3: Calculate E(P\_i P\_j)}}\label{step-3-calculate-ep_i-p_j}

We know:

\[ E(P_i P_j) = E((G_i + e_i)(G_j + e_j)) \]
\[ = E(G_i G_j) + E(G_i e_j) + E(e_i G_j) + E(e_i e_j) \]

Given that \(G_i\) and \(e_i\) are independent and have zero means,
\(E(G_i e_j) = E(e_i G_j) = 0\). If \(e_i\) and \(e_j\) are also
independent, \(E(e_i e_j) = 0\). Thus:

\[ E(P_i P_j) = E(G_i G_j) \]

\paragraph{\texorpdfstring{Step 4: Total Phenotypic Variance
\(V(P)\)}{Step 4: Total Phenotypic Variance V(P)}}\label{step-4-total-phenotypic-variance-vp}

\[ V(P) = E(P^2) - (E(P))^2 \]

Since \(E(P) = 0\):

\[ V(P) = E(P^2) \]
\[ = \frac{1}{n^2} \sum_{i=1}^n \sum_{j=1}^n E(P_i P_j) \]
\[ = \frac{1}{n^2} \left( \sum_{i=1}^n E(P_i^2) + \sum_{i \neq j} E(P_i P_j) \right) \]

From earlier, we have:

\[ E(P_i^2) = \text{Var}(P_i) = \text{Var}(G_i) + \text{Var}(e_i) \]
\[ E(P_i P_j) = \text{Cov}(G_i, G_j) \]

Thus:

\[ V(P) = \frac{1}{n^2} \left( n (\text{Var}(G) + \text{Var}(e)) + n(n-1) \text{Cov}(G_i, G_j) \right) \]
\[ = \frac{1}{n} (\text{Var}(G) + \text{Var}(e)) + \frac{n-1}{n} \text{Cov}(G_i, G_j) \]

Given that we typically separate the genetic and environmental variances
and their covariances, we can simplify this to:

\[ V(P) = \text{Var}(G) + \text{Var}(e) + \text{Cov}(G_i, G_j) \]

\subsubsection{Conclusion}\label{conclusion}

The total phenotypic variance \(V(P)\) is:

\[ V(P) = V_G + V_e + \text{Cov}(G_i, G_j) \]

This expression correctly separates the genetic variance (\(V_G\)), the
environmental variance (\(V_e\)), and the covariance between the genetic
effects of different individuals (\(\text{Cov}(G_i, G_j)\)). This
approach ensures that all sources of variation and their interactions
within the population are accounted for. \#\# What happens if Phenotypes
are not independent

Now write E(Pi - 1/n)

consider the covariance between two phenotypes \(P_i\) and \(P_j\):

\[ \text{Cov}(P_i, P_j) = E[P_i P_j] - E[P_i]E[P_j] \]

Given that \(E(P_i) = 0\) and \(E(P_j) = 0\), we have:

\[ \text{Cov}(P_i, P_j) = E[P_i P_j] \]

Let's expand \(P_i P_j\):

\[ P_i P_j = (DGE_i + e_i)(DGE_j + e_j) = DGE_i DGE_j + DGE_i e_j + e_i DGE_j + e_i e_j \]

Taking the expectation:

\[ E[P_i P_j] = E[DGE_i DGE_j] + E[DGE_i e_j] + E[e_i DGE_j] + E[e_i e_j] \]

Again, since \(DGE\) and \(e\) are independent,
\(E[DGE_i e_j] = E[e_i DGE_j] = 0\). If \(e_i\) and \(e_j\) are also
independent, \(E[e_i e_j] = 0\). Thus:

\[ E[P_i P_j] = E[DGE_i DGE_j] \]

So:

\[ \text{Cov}(P_i, P_j) = \text{Cov}(DGE_i, DGE_j) \]

\subsubsection{Total Phenotypic
Variance}\label{total-phenotypic-variance}

When considering the total phenotypic variance for a population, we need
to include both the variance of individual phenotypes and the
covariances between different phenotypes. The total phenotypic variance
\(V(P)\) is given by:

\[ V(P) = \sum_i \text{Var}(P_i) + \sum_{i \neq j} 2 \cdot \text{Cov}(P_i, P_j) \]

\paragraph{Putting It All Together}\label{putting-it-all-together}

\begin{enumerate}
\def\labelenumi{\arabic{enumi}.}
\item
  \textbf{Variance of \(P_i\)}:
  \[ \text{Var}(P_i) = \text{Var}(DGE_i) + \text{Var}(e_i) \]
\item
  \textbf{Covariance between \(P_i\) and \(P_j\)}:
  \[ \text{Cov}(P_i, P_j) = \text{Cov}(DGE_i, DGE_j) \]
\end{enumerate}

So, the total phenotypic variance \(V(P)\) is:

\[ V(P) = \sum_i \left( \text{Var}(DGE_i) + \text{Var}(e_i) \right) + \sum_{i \neq j} 2 \cdot \text{Cov}(DGE_i, DGE_j) \]

Given that we are taking expectations, the final formula can be written
as:

\[ V(P) = N \left( \mathbb{E}[\text{Var}(DGE)] + \mathbb{E}[\text{Var}(e)] \right) + N(N-1) \cdot 2 \cdot \mathbb{E}[\text{Cov}(DGE_i, DGE_j)] \]

where \(N\) is the number of individuals.

\subsubsection{Conclusion}\label{conclusion-1}

In the classic case, the total phenotypic variance \(V(P)\) includes: -
The variances of the direct genetic effects and environmental effects of
individuals. - The covariance between the direct genetic effects of
different individuals, counted twice because each pairwise interaction
contributes to the total variance.

This thorough approach ensures that we account for all sources of
variation and their interactions within the population.

``\,``\,``\,``\,``\,``\,``\,``\,``\,``\,``\,``\,``\,``\,``\,``\,``\,``\,``\,``\,``\,``\,``\,``\,``\,``\,``\,``\,``\,``\,``\,''

In the case of heterogeneous populations, such as Bulks or landraces or
farmers varieties, individuals of different genotypes are neighbouring
each others. So, if we introduce the fact that the phenotype of a
\(focal\) individual, \(i\), \(P_{Focal_i}\), can be determined by its
own genotype and by the genotype of its neighbour, \(j\),
\(P_{Focal_i, Neighbour_j}\) can now be decomposed into a Direct Genetic
Effect (\(DGE\)) due to \(i\) and the Indirect Genetic Effect (\(IGE\))
from its neighbour \(j\), the model becomes:

\[P_{Focal_i, Neighbour_j} = DGE_{Focal_i} + IGE_{Neignhbour_j} + E_{DGE_{Focal_i}} + E_{IGE_{Neignhbour_j}}\]
Where:\\
- \(DGE_{Focal_i}\) is a Direct Genetic Effect of \(i\), i.e., the
direct contribution of the individual's own genotype to its own
phenotype,\\
- \(IGE_{Neignhbours}\) is the Indirect genetic contribution of the
genotype of the neighbourc \(j\)\\
and,\\
- \(E_{DGE_{Focal_i}}\) : environment effect on Direct Genetic Effect)
on the focal plant\\
- \(E_{IGE_{Neignhbour_j}}\) :environment effect on Indirect Genetic
Effect of the neighbour

These components have their associated variances :
\(V_{DGE},V_{IGE}, V_{E_{DGE}}, V_{E_{IGE}}\)

Of course, any individual in an heterogeneous populations, if a
individual \(i\) has a \(DGE_i\) it also have a \(IGE_i\) than can be
correlated.

\subsection{Correlation between DGE and
IGE}\label{correlation-between-dge-and-ige}

A correlation between \(DGE_i\) and \(IGE_i\) will affects the overall
distribution of the phenotypic values and hence the response to
selection. For a social trait, a strong competitor, \(i\), can have a
strong positive \(DGE_i\) value while reducing the trait value of its
neighbours with a negative \(IGE_i\) value.

According to the trait, the correlation between the \(DGE_i\) and their
\(IGE_i\) counterpart can theoretically vary from -1 to 1. This
correlation will thus translate in the data into a covariance between
the \(DGE_i\) and \(IGE_i\) values, denoted \(cov(DGE:IGE)\) with

\[r_{(DGE:IGE)}=\frac{cov(DGE:IGE)}{\sqrt{V_{DGE}V_{IGE}}}\]

\subsection{Phenotypic variance in a complex
neighbourhood}\label{phenotypic-variance-in-a-complex-neighbourhood}

In our workshop situation, we will consider that plants are grown on a
grid and any of them have 8 neighbours with which it is interacting.

\begin{longtable}[]{@{}lll@{}}
\toprule\noalign{}
\endhead
\bottomrule\noalign{}
\endlastfoot
Neighbour1 & Neighbour2 & Neighbour3 \\
Neighbour4 & Focal & Neighbour5 \\
Neighbour6 & Neighbour7 & Neighbour8 \\
\end{longtable}

\[ \begin{split}
P_{i, (n1..n8)} = & DGE_{i} \,+ 
                        \\ & IGE_{n1} \, +..+ \,IGE_{n8} \,+ 
                        \\& E_{DGE_{i}} \,+ 
                        \\& E_{IGE_{n1}}+..+E_{IGE_{n8}}
\end{split}\]

To calculate the total phenotypic variance \(V_P\), we may recall the
usual case where \(P=G+E\) when the phenotypic values can be considered
as independent when the genotypes are not related and when there is no
particular structured environment. In this case \$V\_P = V\_P + \$

If a mass selection is performed based on phenotypic value, the
selection acts on both the individual's genotype and the genotypes of
its neighbors, a concept known as the social phenotype.

\subparagraph{True Breeding Value (TBV)}\label{true-breeding-value-tbv}

The true breeding value (TBV) integrates both DGE and IGE. It is an
important metric for predicting the genetic value of the next
generation, particularly when social interactions are considered. The
TBV is influenced by the correlation between DGE and IGE and can be
calculated as: \[ TBV = DGE + \sum IGE \] where \(\sum IGE\) represents
the sum of the indirect genetic effects from all relevant neighbors.

\subparagraph{Environmental Effects}\label{environmental-effects}

Environmental effects (\(E\)) play a significant role in shaping the
phenotypic value. These effects can be unique to each individual
(specific environment) or shared among individuals (common environment).
In our model, we consider environmental variances for both DGE and IGE.

\paragraph{Bibliographic Context}\label{bibliographic-context}

Understanding the contributions of genetic and environmental factors to
phenotypic variance has been a core pursuit in quantitative genetics.
The inclusion of social interactions, as captured by IGE, has been
pivotal in advancing breeding programs, particularly in plants and
animals with significant social structures. Studies (ref) have laid the
foundation for integrating IGEs into genetic models, highlighting their
impact on selection response and breeding strategies.

\paragraph{Simulation Algorithm}\label{simulation-algorithm}

The simulation algorithm used in this practical session involves the
following steps:

\begin{enumerate}
\def\labelenumi{\arabic{enumi}.}
\tightlist
\item
  \textbf{Matrix Preparation}: Genetic (G) and environmental (E)
  variance-covariance matrices are prepared based on user inputs.
\item
  \textbf{Genotype and Neighborhood Generation}: Genotypes and
  environmental effects are generated for a grid of plants, and
  neighbors are assigned randomly.
\item
  \textbf{Phenotypic Value Calculation}: The phenotypic value for each
  plant is calculated as the sum of its DGE, the IGEs from its
  neighbors, and environmental effects.
\item
  \textbf{Selection}: A proportion of the top-performing plants is
  selected based on their phenotypic values.
\item
  \textbf{Next Generation Prediction}: The mean genetic values for the
  next generation are predicted based on the selected plants, taking
  into account the variances and covariance between DGE and IGE.
\end{enumerate}

\paragraph{Detailed Steps in the
Simulation}\label{detailed-steps-in-the-simulation}

\begin{enumerate}
\def\labelenumi{\arabic{enumi}.}
\tightlist
\item
  \textbf{Matrix Preparation}:

  \begin{itemize}
  \tightlist
  \item
    Calculate the covariance between DGE and IGE based on the
    user-provided genetic correlation.
  \item
    Form the genetic variance-covariance matrix \(G\) and the
    environmental variance-covariance matrix \(E\).
  \end{itemize}

\begin{Shaded}
\begin{Highlighting}[]
\NormalTok{covdgE\_IGE }\OtherTok{\textless{}{-}}\NormalTok{ input}\SpecialCharTok{$}\NormalTok{r }\SpecialCharTok{*}\NormalTok{ (input}\SpecialCharTok{$}\NormalTok{varG11 }\SpecialCharTok{*}\NormalTok{ input}\SpecialCharTok{$}\NormalTok{varG22)}\SpecialCharTok{\^{}}\FloatTok{0.5}
\NormalTok{G }\OtherTok{\textless{}{-}} \FunctionTok{matrix}\NormalTok{(}\FunctionTok{c}\NormalTok{(input}\SpecialCharTok{$}\NormalTok{varG11, covdgE\_IGE, covdgE\_IGE, input}\SpecialCharTok{$}\NormalTok{varG22), }\AttributeTok{nrow =} \DecValTok{2}\NormalTok{, }\AttributeTok{byrow =} \ConstantTok{TRUE}\NormalTok{)}
\NormalTok{E }\OtherTok{\textless{}{-}} \FunctionTok{matrix}\NormalTok{(}\FunctionTok{c}\NormalTok{(input}\SpecialCharTok{$}\NormalTok{varE11, }\DecValTok{0}\NormalTok{, }\DecValTok{0}\NormalTok{, input}\SpecialCharTok{$}\NormalTok{varE22), }\AttributeTok{nrow =} \DecValTok{2}\NormalTok{, }\AttributeTok{byrow =} \ConstantTok{TRUE}\NormalTok{)}
\end{Highlighting}
\end{Shaded}
\item
  \textbf{Genotype and Neighborhood Generation}:

  \begin{itemize}
  \tightlist
  \item
    Generate genotype values using a multivariate normal distribution
    for a grid of \(N \times N\) plants.
  \item
    Generate environmental effects similarly.
  \item
    Assign neighbors randomly on a grid using a neighborhood matrix.
  \end{itemize}

\begin{Shaded}
\begin{Highlighting}[]
\NormalTok{genotype }\OtherTok{\textless{}{-}} \FunctionTok{mvrnorm}\NormalTok{(}\AttributeTok{n =}\NormalTok{ N}\SpecialCharTok{\^{}}\DecValTok{2}\NormalTok{, }\AttributeTok{mu =} \FunctionTok{c}\NormalTok{(}\DecValTok{0}\NormalTok{, }\DecValTok{0}\NormalTok{), }\AttributeTok{Sigma =}\NormalTok{ G)}
\NormalTok{enviro }\OtherTok{\textless{}{-}} \FunctionTok{mvrnorm}\NormalTok{(}\AttributeTok{n =}\NormalTok{ N}\SpecialCharTok{\^{}}\DecValTok{2}\NormalTok{, }\AttributeTok{mu =} \FunctionTok{c}\NormalTok{(}\DecValTok{0}\NormalTok{, }\DecValTok{0}\NormalTok{), }\AttributeTok{Sigma =}\NormalTok{ E)}
\NormalTok{voisinage }\OtherTok{\textless{}{-}} \FunctionTok{matrix}\NormalTok{(}\FunctionTok{sample}\NormalTok{(}\DecValTok{1}\SpecialCharTok{:}\NormalTok{N}\SpecialCharTok{\^{}}\DecValTok{2}\NormalTok{, }\DecValTok{4}\SpecialCharTok{*}\NormalTok{N}\SpecialCharTok{\^{}}\DecValTok{2}\NormalTok{, }\AttributeTok{replace =} \ConstantTok{TRUE}\NormalTok{), }\AttributeTok{nrow =} \DecValTok{2}\SpecialCharTok{*}\NormalTok{N, }\AttributeTok{ncol =} \DecValTok{2}\SpecialCharTok{*}\NormalTok{N)}
\end{Highlighting}
\end{Shaded}
\item
  \textbf{Phenotypic Value Calculation}:

  \begin{itemize}
  \tightlist
  \item
    For each plant, calculate its DGE and sum the IGEs from its
    neighbors.
  \item
    Add environmental effects to obtain the total phenotypic value.
  \end{itemize}

\begin{Shaded}
\begin{Highlighting}[]
\NormalTok{Valeur\_G }\OtherTok{\textless{}{-}} \ControlFlowTok{function}\NormalTok{(x, y, genotype, voisinage) \{}
\NormalTok{    geno }\OtherTok{\textless{}{-}} \FunctionTok{c}\NormalTok{()}
    \ControlFlowTok{for}\NormalTok{ (i }\ControlFlowTok{in}\NormalTok{ (x}\DecValTok{{-}1}\NormalTok{)}\SpecialCharTok{:}\NormalTok{(x}\SpecialCharTok{+}\DecValTok{1}\NormalTok{)) \{}
        \ControlFlowTok{for}\NormalTok{ (j }\ControlFlowTok{in}\NormalTok{ (y}\DecValTok{{-}1}\NormalTok{)}\SpecialCharTok{:}\NormalTok{(y}\SpecialCharTok{+}\DecValTok{1}\NormalTok{)) \{}
            \ControlFlowTok{if}\NormalTok{ (}\SpecialCharTok{!}\NormalTok{(i }\SpecialCharTok{==}\NormalTok{ x }\SpecialCharTok{\&\&}\NormalTok{ j }\SpecialCharTok{==}\NormalTok{ y)) \{}
\NormalTok{                geno }\OtherTok{\textless{}{-}} \FunctionTok{c}\NormalTok{(geno, voisinage[i, j])}
\NormalTok{            \}}
\NormalTok{        \}}
\NormalTok{    \}}
\NormalTok{    DGE }\OtherTok{\textless{}{-}}\NormalTok{ genotype[voisinage[x, y], }\DecValTok{1}\NormalTok{] }\SpecialCharTok{+}\NormalTok{ enviro[voisinage[x, y], }\DecValTok{1}\NormalTok{]}
\NormalTok{    IGE }\OtherTok{\textless{}{-}} \FunctionTok{sum}\NormalTok{(genotype[geno, }\DecValTok{2}\NormalTok{] }\SpecialCharTok{+}\NormalTok{ enviro[geno, }\DecValTok{2}\NormalTok{])}
    \FunctionTok{return}\NormalTok{(DGE }\SpecialCharTok{+}\NormalTok{ IGE)}
\NormalTok{\}}

\NormalTok{Pheno }\OtherTok{\textless{}{-}} \FunctionTok{c}\NormalTok{()}
\NormalTok{tri\_geno }\OtherTok{\textless{}{-}} \FunctionTok{c}\NormalTok{()}
\NormalTok{cpt }\OtherTok{\textless{}{-}} \DecValTok{0}

\ControlFlowTok{for}\NormalTok{ (x }\ControlFlowTok{in} \DecValTok{5}\SpecialCharTok{:}\NormalTok{(N}\SpecialCharTok{+}\DecValTok{5}\NormalTok{)) \{}
    \ControlFlowTok{for}\NormalTok{ (y }\ControlFlowTok{in} \DecValTok{5}\SpecialCharTok{:}\NormalTok{(N}\SpecialCharTok{+}\DecValTok{5}\NormalTok{)) \{}
\NormalTok{        cpt }\OtherTok{\textless{}{-}}\NormalTok{ cpt }\SpecialCharTok{+} \DecValTok{1}
\NormalTok{        Pheno[cpt] }\OtherTok{\textless{}{-}} \FunctionTok{Valeur\_G}\NormalTok{(x, y, genotype, voisinage)}
\NormalTok{        tri\_geno[cpt] }\OtherTok{\textless{}{-}}\NormalTok{ voisinage[x, y]}
\NormalTok{    \}}
\NormalTok{\}}
\end{Highlighting}
\end{Shaded}
\item
  \textbf{Selection}:

  \begin{itemize}
  \tightlist
  \item
    Rank individuals by their phenotypic values.
  \item
    Select the top proportion (\(p\)) of individuals.
  \end{itemize}

\begin{Shaded}
\begin{Highlighting}[]
\NormalTok{N\_sel }\OtherTok{\textless{}{-}} \FunctionTok{round}\NormalTok{(cpt }\SpecialCharTok{*}\NormalTok{ input}\SpecialCharTok{$}\NormalTok{p)}
\NormalTok{geno\_sel }\OtherTok{\textless{}{-}}\NormalTok{ genotype[tri\_geno[}\FunctionTok{order}\NormalTok{(Pheno, }\AttributeTok{decreasing =} \ConstantTok{TRUE}\NormalTok{)][}\DecValTok{1}\SpecialCharTok{:}\NormalTok{N\_sel], ]}
\end{Highlighting}
\end{Shaded}
\item
  \textbf{Next Generation Prediction}:

  \begin{itemize}
  \tightlist
  \item
    Calculate the mean genetic values for the next generation based on
    the selected individuals.
  \item
    Adjust the mean values considering the response to selection,
    including the correlation between DGE and IGE.
  \end{itemize}

\begin{Shaded}
\begin{Highlighting}[]
\NormalTok{S }\OtherTok{\textless{}{-}} \FunctionTok{c}\NormalTok{()}
\NormalTok{S[}\DecValTok{1}\NormalTok{] }\OtherTok{\textless{}{-}} \FunctionTok{mean}\NormalTok{(genotype[tri\_geno[}\FunctionTok{order}\NormalTok{(Pheno, }\AttributeTok{decreasing =} \ConstantTok{TRUE}\NormalTok{)][}\DecValTok{1}\SpecialCharTok{:}\NormalTok{N\_sel], }\DecValTok{1}\NormalTok{])}
\NormalTok{S[}\DecValTok{2}\NormalTok{] }\OtherTok{\textless{}{-}} \FunctionTok{mean}\NormalTok{(genotype[tri\_geno[}\FunctionTok{order}\NormalTok{(Pheno, }\AttributeTok{decreasing =} \ConstantTok{TRUE}\NormalTok{)][}\DecValTok{1}\SpecialCharTok{:}\NormalTok{N\_sel], }\DecValTok{2}\NormalTok{])}

\NormalTok{R }\OtherTok{\textless{}{-}}\NormalTok{ G }\SpecialCharTok{\%*\%} \FunctionTok{solve}\NormalTok{(P) }\SpecialCharTok{\%*\%}\NormalTok{ S}
\NormalTok{mus }\OtherTok{\textless{}{-}}\NormalTok{ mu}
\NormalTok{mus[}\DecValTok{1}\NormalTok{] }\OtherTok{\textless{}{-}}\NormalTok{ mu[}\DecValTok{1}\NormalTok{] }\SpecialCharTok{+}\NormalTok{ R[}\DecValTok{1}\NormalTok{]}
\NormalTok{mus[}\DecValTok{2}\NormalTok{] }\OtherTok{\textless{}{-}}\NormalTok{ mu[}\DecValTok{2}\NormalTok{] }\SpecialCharTok{+}\NormalTok{ R[}\DecValTok{2}\NormalTok{]}

\NormalTok{new\_geno }\OtherTok{\textless{}{-}} \FunctionTok{mvrnorm}\NormalTok{(}\AttributeTok{n =}\NormalTok{ N}\SpecialCharTok{\^{}}\DecValTok{2}\NormalTok{, }\AttributeTok{mu =}\NormalTok{ mus, }\AttributeTok{Sigma =}\NormalTok{ G)}
\NormalTok{new\_voisinage }\OtherTok{\textless{}{-}} \FunctionTok{matrix}\NormalTok{(}\FunctionTok{sample}\NormalTok{(}\DecValTok{1}\SpecialCharTok{:}\NormalTok{N}\SpecialCharTok{\^{}}\DecValTok{2}\NormalTok{, }\DecValTok{4}\SpecialCharTok{*}\NormalTok{N}\SpecialCharTok{\^{}}\DecValTok{2}\NormalTok{, }\AttributeTok{replace =} \ConstantTok{TRUE}\NormalTok{), }\AttributeTok{nrow =} \DecValTok{2}\SpecialCharTok{*}\NormalTok{N, }\AttributeTok{ncol =} \DecValTok{2}\SpecialCharTok{*}\NormalTok{N)}

\NormalTok{new\_Pheno }\OtherTok{\textless{}{-}} \FunctionTok{c}\NormalTok{()}
\NormalTok{new\_tri\_geno }\OtherTok{\textless{}{-}} \FunctionTok{c}\NormalTok{()}
\NormalTok{cpt }\OtherTok{\textless{}{-}} \DecValTok{0}

\ControlFlowTok{for}\NormalTok{ (x }\ControlFlowTok{in} \DecValTok{5}\SpecialCharTok{:}\NormalTok{(N}\SpecialCharTok{+}\DecValTok{5}\NormalTok{)) \{}
    \ControlFlowTok{for}\NormalTok{ (y }\ControlFlowTok{in} \DecValTok{5}\SpecialCharTok{:}\NormalTok{(N}\SpecialCharTok{+}\DecValTok{5}\NormalTok{)) \{}
\NormalTok{        cpt }\OtherTok{\textless{}{-}}\NormalTok{ cpt }\SpecialCharTok{+} \DecValTok{1}
\NormalTok{        new\_Pheno[cpt] }\OtherTok{\textless{}{-}} \FunctionTok{Valeur\_G}\NormalTok{(x, y, new\_geno, new\_voisinage)}
\NormalTok{        new\_tri\_geno[cpt] }\OtherTok{\textless{}{-}}\NormalTok{ new\_voisinage[x, y]}
\NormalTok{    \}}
\NormalTok{\}}
\end{Highlighting}
\end{Shaded}
\end{enumerate}

By understanding these steps and using the Shiny application, you will
gain insights into the dynamics of genetic selection and the impact of
social interactions on breeding outcomes.

\paragraph{Shiny Application
Instructions}\label{shiny-application-instructions}

To use the Shiny application for this practical session, follow these
steps:

\begin{enumerate}
\def\labelenumi{\arabic{enumi}.}
\tightlist
\item
  Open the Shiny application in your web browser.
\item
  Set the number of individuals (N) using the slider.
\item
  Adjust the genetic variances for DGE and IGE, the genetic correlation,
  and the environmental variances using the provided inputs.
\item
  Set the selection pressure to determine the proportion of
  top-performing plants to be selected.
\item
  If desired, set a seed for reproducibility by checking the box and
  entering a seed value.
\item
  Click the ``Run Simulation'' button to start the simulation.
\item
  View the results in the ``Graph'' and ``Summary'' tabs, which show the
  genotype distribution, phenotypic distributions before and after
  selection, and a summary of the calculations.
\end{enumerate}

\paragraph{Suggested Parameters}\label{suggested-parameters}

To help you understand the effects of selection, try using the following
parameter values:

\begin{itemize}
\tightlist
\item
  \textbf{Number of individuals (N)}: 20
\item
  \textbf{Genetic variance DGE (\(V_{DGE}\))}: 0.75
\item
  \textbf{Genetic variance IGE (\(V_{IGE}\))}: 0.07
\item
  \textbf{Genetic correlation DGE : IGE (\(r\))}: 0.5
\item
  **Environmental variance DGE
\end{itemize}

(\(V_{E_{DGE}}\))\textbf{: 0.7 - }Environmental variance IGE
(\(V_{E_{IGE}}\))\textbf{: 0.07 - }Selection pressure (\(p\))**: 0.3

Observe how changes in these parameters affect the selection outcomes
and the predicted genetic values of the next generation.

\end{document}
